\chapter{Úvod}
Každý databázový systém má svoj dotazovací jazyk pomocou ktorého s ním užívateľ môže manipulovať. Tento jazyk je jasne definovaný svoju syntaxou, ktorá určuje súhrn pravidiel udávajúcich prípustné tvary čiastkových konštrukcií a celého dotazu. Nato aby databázový systém vedel, akú akciu sa snaží užívateľ vykonať musí zanalyzovať výraz napísaný pomocou jeho syntaxe.

Projekt Debezium si v pamäti udržiava štruktúru sledovanej databáze a snaží sa zachytávať zmeny nad touto databázou. Rovnako ako databázový systém musí zanalyzovať syntaxi spusteného výrazu, aby vedel, ako užívateľ mení štruktúru databáze a mohol rovnaké zmeny aplikovať na svoj model uložený v pamäti.
Stávajúci syntaktický analyzátor je ručne napísaný, veľmi jednoduchý a zďaleka nepostihuje všetky nuance \nom{SQL}{Structured Query Language} jazyka, čím sa stáva náchylným k chybám. Novo implementovaný strojovo generovaný syntaktický analyzátor nahradí aktuálne riešenie v projekte Debezium, čím sa zníži pravdepodobnosť vzniku chýb, a bude možné ho upraviť jednoduchou zmenou v gramatike jazyka nad ktorým bude pracovať.

\section{Motivácia}
Analýza MySQL DDL príkazov alebo akejkoľvek inej dôležitej relačnej databázy sa môže javiť ako skľučujúca úloha. Zvyčajne databázový systém má vysoko prispôsobenú gramatiku SQL a hoci výrazy jazyka manipulácie s údajmi (DML) sú často pomerne blízke štandardom, výrazy jazyka pre definíciu dát (DDL) sú zvyčajne menej a zahŕňajú viac špecifických funkcií databázového systému.

Mnoho aktuálne implementovaných a prístupných analyzátorov rieši iba analýzu základných DDL výrazov a nepodporuje špecifické možnosti jednotlivých databázových systémov. Existujú aj analyzátory napísané konkrétne pre MySQL, no často sú nekompletné  alebo nepodporujú poslednú verziu tejto databáze. Použitie týchto dostupných, no nekompletných implementácií by mohlo pokryť väčšinu požiadavkou projektu Debezium, no zvyšok by musel byť implementovaný iným spôsobom, čo by bolo veľmi zmätočné a náchylné k chybám.

\section{Cieľ práce}
Cieľom práce je zanalyzovať projekt Debezim, jeho aktuálnu implementáciu MySQL DDL syntaktického analyzátoru a navrhnúť nové riešenie. Implementácia nového riešenia by mala byť intuitívne pochopiteľná, bez nutnosti študovania rozsiahlych dokumentačných materiálov. Projekt Debezium plánuje v budúcnosti rozširovať množstvo podporovaných databázových systémov, a preto by výsledok tejto práce mal byť implementovaný čo najprijateľnejšie voči jeho potencionálnemu prepoužitiu pri syntaktických analyzátoroch iných databázových systémov.