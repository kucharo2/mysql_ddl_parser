\chapter{Záver}
Cieľom tejto práce bolo zanalyzovať možnosti nahradenia existujúceho DDL parseru pre MySQL, ktorý v projekte Debezium odchytáva zmeny v databázoých štruktúrach. Aktuálne riešenie nebolo dostačujúce na spracovanie všetkých možných nuancí jazyka MySQL a náročnosť prípadných úprav rástla z veľkosťou jeho implementácie. V rámci analýzy existujúceho parsru boli objavené nekonzistencie voči MySQL syntaxi, čo znamená, že parser príjmal aj nevalidné SQL dotazy.

Analýzou možností, ako by stávajúci parser mohol byť nahradený sa ukázalo, že parsovanie je teoreticky vyriešený problém, no v praxi sa tento problém stále znovu rieši. To znamená, že existuje vela rôznych algoritmov, každý so silnými a slabými stránkami a stále sa vylepšujú. Pre správnu voľbu parsovacieho algoritmu existuje niekoľko faktorov, ktoré je potrebné brať do úvahy. Najdôležitejším z nich je uvedomiť si, do akého jazyka spadá gramatika, ktorú sa snažíme parsovať. Jazyk MySQL spadá pod bezkontextové jazyky, nakoľko obsahuje možnosti rekurzívnych pravidiel. Vačšina algoritmov, ktorá dokáže parsovať bezkontextové jazyky vyžaduje úpravu gramatiky, ako napríklad odstránenie ľavých rekurzií u LL algoritmov. 

Vďaka neustálemu rozširovaniu možnosí parsovania už existujú nástroje, ktoré dokážu niektoré z týchto úprav vykonať, takže autor gramatiky ich môže ignorovať. Nástroj ANTLR verzie 4, ktorý som si zvolil pre riešenie nového parseru, používa aktuálne najnovší parsovací algoritmus ALL(*), ktorý je postavený na LL algoritmoch a bol vyvinutý v roku 2014 práve autorom nástroja ANTLR. Pri generovaní parseru z gramatiky sa ANTLR dokáže vysporiadať s pravidlami obsahujúcimi priamu ľavú rekurziu, no autor si stále musí dávať pozor na nepriamu ľavú rekurziu.

Pri návrhu implementácie nového parseru, generovaného nástrojom ANTLR sa taktiež ukázali nedostatky v existujúcom návrhu, ktorý nebol dostatočne pripravený na možnosti rozšírenia parsovania iným spôsobom. Z toho dôvodu bolo nutné upraviť aktuálny návrh tak, aby bolo možné využiť čo najväčšiu časť existujúcej implementácie. Projekt Debezium bude v budúcnosti rozširovať podporu databázových systémov, ktoré bude možné sledovať, a preto sa pri návrhu počítalo z použitím ANTLR nástroja pri parsovaní iných DBMS. Základná implementácia, ktorá by mala byť spoločná pre všetky ANTLR parsre je implementovaná v samostatnom module, ktorý sa taktiež stará o generovanie parserov z daných gramatík.

Funkcionalita implementácie nového parseru bola úspešne overená testvacou sadou Debezia. V niektorých prípadoch bolo potrebné túto sadu upraviť a to najmä zo spomínaného dôvodu, že predchádzajúca implementácia povoľovala parsovanie syntakticky nevalidných SQL dotazov. V rámci nového ANTLR parseru boli taktiež implementované opravy nájdených chýb a parsovanie SQL dotazov, ktoré v bývalej implementácii neboli parsované. Na základe týchto novoparsovaných dotazov bola existujúca sada rozšírená. Súčasťou novej sady je taktiež kontrola správnosti gramatiky, ktorá môže byť v budúcnosti upravovaná.

V tejto práci sa mi podarilo zanalyzovať možnosti implementácie generovaného parseru a taktiež takýto parser implementovať pre budúce potreby projektu Debezium. Táto implementácia prináša veľa výhod pre projekt Debezium a to najmä v zmysle údržby parseru a prípadnej implementácie parseru pre iný databázový systém. Analýza jednotlivých sparsovaných dotazov je rozdelená do viacerých tried, čo prináša väčšiu prehľadnosť a jednoduchšiu orientáciu pre vývojarov, ktorí s DDL parserom prídu do styku.